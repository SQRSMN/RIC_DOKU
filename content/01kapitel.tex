%!TEX root = ../dokumentation.tex

\chapter{Problemstellung}
In den meisten Projekten die sich mit autonomen Systemen befassen ist die Intelligenz sehr stark an den Roboter bzw. die Basisplattform gebunden. Die Aufgabe dieses Projektes soll es sein erste Versuche darin zu unternehmen eine physikalische Trennung zwischen Intelligenz und mobiler Plattform zu erreichen. Genauer soll die Steuerung des Roboters in eine periphere Infrastruktur verlagert werden.\\
Dadurch entstehen folgende Aufgabenbereiche:
\begin{itemize}
\item Die Kommunikation zwischen den beteiligten Elementen mobile Plattform, intelligente Infrastruktur, Benutzer sicherstellen.
\item Die Integration des Parallelprojekts von Daniel Geiger TINF13B2
\item Sinnvolle Architektur der Infrastruktur
\item Integration Erfassender Komponenten in die Infrastruktur
\item Basis für die Wegfindung der mobilen Plattform
\item Erkennen von Hindernissen innerhalb des Bewegungsraums
\item Reaktion auf erfasste Hindernisse
\item Identifikation und Ortung der Plattform im Bewegungsraum
\end{itemize}
\section{Erwartetes Ergebnis}
Das erwartete Ergebnis dieses Projekts ist dass sich die mobile Plattform innerhalb einer begrenzten und definierten Umgebung autonom bewegen kann. Sie reagiert selbstständig auf auftauchende Hindernisse. Das autonome Bewegen soll durch einen benutzerseitigen Trigger initiiert werden.
