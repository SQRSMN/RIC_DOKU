%!TEX root = ../dokumentation.tex
\chapter{Hindernisse}
	\section{Erkennen von Hindernissen}
		\begin{itemize}
		\item opencv
		\item cv-bridge
		\item ros eigene sensor-msgs\\images Nachrichten in OpenCV format bringen
		\item wo liegt der Unterschied?
		\end{itemize}
		\subsection{Objekte erkennen}
		\subsection{Objekte Identifizieren}
	\section{Reaktion auf Hindernisse}
		\subsection{Grundsätzliche Verhaltensweise bei auftauchenden Hindernissen}
		\begin{itemize}
		\item Hinderniss taucht im Bewegungsraum auf
			\begin{itemize}
				\item Stop
				\item Bewegungsanalyse
				\item Hindernis kommt direkt auf Roboter zu -> weiterhin stehen bleiben
				\item Hindernis stoppt und bewegt sich nicht mehr weiter -> Umgehung bestimmen, umfahren
				\item Hindernis stoppt und bewegt sich weiter -> Roboter bleibt solange stehen bis Hindernis den Bewegungsraum verlassen hat oder sich nicht mehr weiter im Raum bewegt(stillstand) 
			\end{itemize}
		\end{itemize}
		\subsection{Kategorisieren von identifizierten Objekten}
			\subsubsection{fortwährend bewegende Objekte}
			\subsubsection{stillstehende Objekte}