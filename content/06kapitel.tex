%!TEX root = ../dokumentation.tex

	\chapter{Hindernisse}
	Die Interaktion mit seiner Umwelt ist ein essentieller Teil in der Anwendung autonomer Systeme. Besonders der Sicherheitsaspekt spielt hier eine primäre Rolle. Die Akzeptanz der potentiellen Nutzer wäre stark beeinträchtigt wenn die Unversehrtheit von umstehenden Lebewesen und Gegenständen nicht sichergestellt werden kann. Die Komplexität bei Umweltinteraktion zeigt sich besonders beim Auftreten dynamischer Events. Ein Beispiel hierfür ist das Erscheinen von Hindernissen im Bewegungsraum einer mobilen Plattform. Um auf solche Ereignisse angemessen zu reagieren müssen dynamische Objekte erkannt, verfolgt und identifiziert werden. Bei der Entkopptlung der Intelligenz vom Roboter selbst, eröffnet sich ein weiteres Problem. Die Unterscheidung zwischen einem dynamischen Objekt und der mobilen Plattform. Im folgenden \underline{Kapitel} soll ein Konzept für ein solches Szenario erarbeitet und umgesetzt werden.
%	\begin{itemize}
%	\item interaktion mit Umwelt essentiell für autonomes System
%	\item Sicherheitsaspekt kein Verletzungsrisiko für umstehende Lebewesen durch Einsatz eines autonomen systems
%	\item Wo besteht hier die Schwierigkeit
%	\item Objekte erkennen
%	\item Objekte verfolgen
%	\item Objekte identifizieren
%	\end{itemize}
		\section{Definition von Hinderniss}
		Die Definition lautet wie folgt: \"Etwas, was das direkte Erreichen eines Ziels, das Weiterkommen be- oder verhindert.\". \cite{duden-hinderniss} Dabei wird offengelassen welche Ursache das Hinderniss sein kann, Gegenstände und Lebewesen werden hier nicht unterschieden, somit wird im weiteren Verlauf lediglich die neutrale Form Objekt genutzt. Genauer von Objekten welche den Bewegungsraum der Mobilen Plattform betreten bzw. sich bereits darin befinden.
		\begin{itemize}
		\item Ein Objekt was sich im Bewegungsraum der Mobilen Plattform befindet.
		\item Unterscheidung unterschiedlicher Hindernisse bzw Objekte durch Klassifizierung.
		\item Notwendig da nicht auf jede Situation gleich reagiert werden kann = Fallunterscheidung. Klassifizierung Werkzeug für Fallunterscheidung.
		\end{itemize}
		
		\begin{itemize}
		\item dynamisches Objekt = Lebewesen oder Gegenstand welches den Bildbereich der Kamera betritt. Bewegt sich weiterhin auf einer festen Achse durch den Bewegungsraum der Plattform.
		\item statisches Objekt = Lebewesen oder Gegenstand welches fest im Bewegungsraum der Plattform verweilt.
		\item chaotisches Objekt = Lebewesen oder Gegenstand welches ohne vorhersehbares Muster im Bewegungsraum der Plattform umherwandelt, und in unregelmäßigen abständen für eine undefinierbare Zeit an einem fixen Punkt verweilt.
		\item \underline{Blitz} Objekt = Lebewesen oder Gegenstand welches den Bewegungsraum der Mobilen Plattform nur sehr flach durchstreift i.e. den Bewegungsraum nicht tief betritt oder ihn lediglich tangiert, aber dennoch von der Raumüberwachung erfasst wird.
		\end{itemize}
		
		
		\section{Objekte erkennen}
					
			\subsection{Methodik}
				\begin{itemize}
				\item was brauch ich dafür?
				\item opencv
				\item cv-bridge
				\item ros eigene sensor-msgs images Nachrichten in OpenCV format bringen
				\item wo liegt der Unterschied?
				\item Installation ROS-Modul
				\item Code erklären -> evtl Diagramm
				\item in CMakelists.txt -> add\_executable + add\_library
				\item anpassungen bezüglich graustufenbilder -> in image\_converter.cpp
				\end{itemize}
				
				\begin{itemize}
				\item Wie funktioniert das mit OpenCV
				\item mit und ohne Farben möglich
				\item Farbe verfolgt Object anhand der Bewegung von Farbwerten in jedem Pixel
				\item findContours
				\item inRange
				\item ohne Farbe
				\item Sequential Images
				\item Pixel-änderungen -> vergleich mit vorgänger Bild -> veränderungen werden gekennzeichnet
				\item absdiff -> braucht graustufe -> subtrahiert bild1 von bild2 und speichert in differenzbild
				
				\end{itemize}
		\section{Objekte Identifizieren}
		\begin{itemize}
		\item lediglich Unterscheidung ob Roboter oder Hindernis
		\item alles was nicht Roboter ist => Hindernis
		\end{itemize}
			\subsection{Unterscheiden zwischen Roboter und Objekt}
			\begin{itemize}
			\item Robotino identifizieren -> Alleinstellungsmerkmal
			\item spezielles Muster? QR-Code? Abstraktes deutliches schwarz/weiß Design
			\item Skelett des Robotino anfertigen?
			\item Wie Perspektivische Unterschiede ausgleichen?
			\end{itemize}
		
	\chapter{Reaktion auf Hindernisse}
		\section{Grundsätzliche Verhaltensweise bei auftauchenden Hindernissen}
		\begin{itemize}
		\item Hinderniss taucht im Bewegungsraum auf
			\begin{itemize}
				\item Stop
				\item Bewegungsanalyse
				\item Hindernis kommt direkt auf Roboter zu -> weiterhin stehen bleiben
				\item Hindernis stoppt und bewegt sich nicht mehr weiter -> Umgehung bestimmen, umfahren
				\item Hindernis stoppt und bewegt sich weiter -> Roboter bleibt solange stehen bis Hindernis den Bewegungsraum verlassen hat oder sich nicht mehr weiter im Raum bewegt(stillstand) 
			\end{itemize}
		\end{itemize}
		\section{Kategorisieren von identifizierten Objekten}
			\subsection{fortwährend bewegende Objekte}
			\subsection{stillstehende Objekte}