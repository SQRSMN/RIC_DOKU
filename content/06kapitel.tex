%!TEX root = ../dokumentation.tex

	\chapter{Hindernisse}
		\section{Definition von Hinderniss}
		
		
		
		\section{Objekte erkennen}
					
			\subsection{Methodik}
				\begin{itemize}
				\item was brauch ich dafür?
				\item opencv
				\item cv-bridge
				\item ros eigene sensor-msgs\\images Nachrichten in OpenCV format bringen
				\item wo liegt der Unterschied?
				\item Installation ROS-Modul
				\item Code erklären -> evtl Diagramm
				\item in CMakelists.txt -> add\_executable + add\_library
				\item anpassungen bezüglich graustufenbilder -> in image\_converter.cpp
				\end{itemize}
				
				\begin{itemize}
				\item Wie funktioniert das mit OpenCV
				\item mit und ohne Farben möglich
				\item Farbe verfolgt Object anhand der Bewegung von Farbwerten in jedem Pixel
				\item findContours
				\item inRange
				\item ohne Farbe
				\item Sequential Images
				\item Pixel-änderungen -> vergleich mit vorgänger Bild -> veränderungen werden gekennzeichnet
				\item absdiff -> braucht graustufe -> subtrahiert bild1 von bild2 und speichert in differenzbild
				
				\end{itemize}
		\section{Objekte Identifizieren}
		\begin{itemize}
		\item lediglich Unterscheidung ob Roboter oder Hindernis
		\item alles was nicht Roboter ist => Hindernis
		\end{itemize}
			\subsection{Unterscheiden zwischen Roboter und Objekt}
			\begin{itemize}
			\item Robotino identifizieren -> Alleinstellungsmerkmal
			\item spezielles Muster? QR-Code? Abstraktes deutliches schwarz/weiß Design
			\item Skelett des Robotino anfertigen?
			\item Wie Perspektivische Unterschiede ausgleichen?
			\end{itemize}
		
	\chapter{Reaktion auf Hindernisse}
		\section{Grundsätzliche Verhaltensweise bei auftauchenden Hindernissen}
		\begin{itemize}
		\item Hinderniss taucht im Bewegungsraum auf
			\begin{itemize}
				\item Stop
				\item Bewegungsanalyse
				\item Hindernis kommt direkt auf Roboter zu -> weiterhin stehen bleiben
				\item Hindernis stoppt und bewegt sich nicht mehr weiter -> Umgehung bestimmen, umfahren
				\item Hindernis stoppt und bewegt sich weiter -> Roboter bleibt solange stehen bis Hindernis den Bewegungsraum verlassen hat oder sich nicht mehr weiter im Raum bewegt(stillstand) 
			\end{itemize}
		\end{itemize}
		\section{Kategorisieren von identifizierten Objekten}
			\subsection{fortwährend bewegende Objekte}
			\subsection{stillstehende Objekte}