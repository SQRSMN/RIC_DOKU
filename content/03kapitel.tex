%!TEX root = ../dokumentation.tex
\chapter{Basis Konzept}
	\begin{itemize}
	\item Grundlegende Idee -> Intelligenz aus Roboter herausnehmen in die Umgebung integrieren
	\item Fokus liegt hier nun auf Kommunikation der beteiligten Komponenten
	\item Grundmechanik auf ROS aufsetzen -> Eventbus -> liefert Basis für Kommunikation und für Eventhandling -> Erweiterungsfreundlich -> sehr gute Doku -> breite Unterstützung

	\item Wie soll die Kommunikation von statten gehen?
	\item ROS-System erklären Netzwerkkommunikation, Eventbus, Nodes Topics

	\item Welche Rolle spielt die Kinect

	\item Wie sieht die geplante Umgebung aus? -> Zeichnung
	\end{itemize}
	\section{Kommunikation}
	\begin{itemize}
	\item ROS nutzt netzwerktechnologien
	\item Medium W-LAN -> Flexibel -> Alle Komponenten einfach um WLAN erweiterbar bzw. schon damit ausgestattet
	\item Kommunikation innerhalb von ROS
	\item ROS nutzt hierfür Messages in verschiedenen einfach zu definierenden formaten
	\end{itemize}
	\section{Eventhandling}
	
	\section{Wahrnehmen der Umgebung}
	
	\section{Feature Konzepte}
	Die Konzepte für die Wegfindung, die Analyse von Hindernissen und die Kommunikation mit der Plattform Robotino befinden sich in den jeweiligen Kapiteln.