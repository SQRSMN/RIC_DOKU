%%**************************************************************
%% Vorlage fuer Bachelorarbeiten (o.ä.) der DHBW
%%
%% Autor: Tobias Dreher, Yves Fischer
%% Datum: 06.07.2011
%%
%% Autor: Michael Gruben
%% Datum: 15.05.2013
%%
%% Autor: Markus Barthel
%% Datum: 22.08.2014
%%**************************************************************

\input{ads/header}

\makeglossaries
\input{ads/glossary}

\begin{document}

	% Deckblatt
	\begin{spacing}{1}
		\input{ads/deckblatt}
	\end{spacing}
	\newpage

	\pagenumbering{Roman}

	% Sperrvermerk
	\input{ads/sperrvermerk}
	\newpage

	% Erklärung
	\input{ads/erklaerung}
	\newpage

	% Abstract
	\input{ads/abstract}
	\newpage

	\pagestyle{plain}		% nur Seitenzahlen im Fuß
	
%	\RedeclareSectionCommand[beforeskip=\kapitelabstand         ]{chapter} % stellt Abstand vor Kapitelüberschriften ein

	% Inhaltsverzeichnis
	\begin{spacing}{1.1}
		\begingroup
		
			% auskommentieren für Seitenzahlen unter Inhaltsverzeichnis
			\renewcommand*{\chapterpagestyle}{empty}
			\pagestyle{empty}
			
			
			\setcounter{tocdepth}{1}
			%für die Anzeige von Unterkapiteln im Inhaltsverzeichnis
			%\setcounter{tocdepth}{2}
			
			\tableofcontents
			\clearpage
		\endgroup
	\end{spacing}
	\newpage

	% Abkürzungsverzeichnis
	\cleardoublepage
	\input{ads/acronyms}

	% Abbildungsverzeichnis
	\cleardoublepage
	\listoffigures

	%Tabellenverzeichnis
	\cleardoublepage
	\listoftables

	% Quellcodeverzeichnis
	\cleardoublepage
	\lstlistoflistings
	\cleardoublepage

	\pagenumbering{arabic}
	
	\pagestyle{headings}		% Kolumnentitel im Kopf, Seitenzahlen im Fuß

	% Inhalt
	\foreach \i in {01,02,03,04,05,06,07,08,09,...,99} {%
		\edef\FileName{content/\i kapitel}%
			\IfFileExists{\FileName}{%
				\input{\FileName}
			}
			{%
				%file does not exist
			}
	}

	\clearpage

	% Literaturverzeichnis

	\printbibliography

	% Glossar
%	\printglossary[style=altlist,title=\langglossar]
	
	% sonstiger Anhang
%	\appendix
%	\input{ads/appendix}

\begin{thebibliography}{9}
\bibitem{kinect-georg} \emph{kinect-georg} \emph{Kinect als Eingabegerät}, Jan Georg.
\bibitem{kinect-hacking} \emph{kinect-hacking} \emph{Hacking the Kinect}, Kramer-Burrus-Echtler-Herrera-Parker, 2012.
\bibitem{alpha-centauri-ueberlicht} \emph{alpha-centauri-ueberlicht} \emph{Alpha Centauri: gibt es Überlichtgeschwindigkeit}, Prof. Harald Lesch, 2004.
\bibitem{cv-filter} \emph{cv-filter} \emph{image filtering}, Georg Lukas, 2002.
\bibitem{loesch} \emph{loesch} \emph{Bayessches Lernen}, Dipl.Inform. Martin L"osch, KIT.
\bibitem{welchowski} \emph{welchowski} \emph{Bayes Netze}, Thomas Welchowski, 2014, Ludwig-Maximilians-Universit"at-M"unchen.
\bibitem{ros-wiki} \emph{ros-wiki} \emph{wiki.ros.org}, ros.org, http://wiki.ros.org/.
\bibitem{ros-intro} \emph{ros-intro} \emph{ROS Introduction}, ros.org, http://wiki.ros.org/ROS/Introduction.
\bibitem{ros-topics} \emph{ros-topics} \emph{ROS topics}, ros.org, http://wiki.ros.org/ROS/Tutorials/UnderstandingTopics.
\bibitem{ros-nodes} \emph{ros-nodes} \emph{ROS nodes}, ros.org, http://wiki.ros.org/ROS/Tutorials/UnderstandingNodes.
\bibitem{find-object-ros} \emph{find-object-ros} \emph{Find-Objects-2D ROS-WIKI}, ros.org, http://wiki.ros.org/find\_object\_2d.
\bibitem{find-object} \emph{ros-nodes} \emph{Find-Objects-2D GitHub}, ros.org, http://introlab.github.io/find-object/.








\end{thebibliography}
	
\end{document}
